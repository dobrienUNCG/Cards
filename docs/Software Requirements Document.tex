\documentclass[12pt]{article}
\usepackage[acronym]{glossaries}
\title{Software Requirements Document}

\newacronym{srd}{SRD}{Software Requirements Document}
\newacronym{jre}{JRE}{Java Runtime Environment}
\newacronym{jdk}{JDK}{Java Development Environment}
\newacronym{jvm}{JVM}{Java Virtual Machin}
\begin{document}
\begin{titlepage}
\begin{center}
\Huge
\textbf{Software Requirements Document}\\
\huge Cards\\
\Large
Copy and Paste Until It Is Done\\
\Large \today\\
\begin{tabular}{|c|}
\hline
Group Members\\
\hline
Devin O'Brien\\
Jake Keels\\
Sage Bonfield\\
\hline
\end{tabular}
\large UNCG Honor Code
\end{center}
\end{titlepage}
\section{Introduction}\label{introduction}

\tableofcontents

\newpage
\subsection{Purpose}

This Software Requirements Document is for the CSC 429-01 Software
Engineering semester project. This document will cover the requirements
that are imposed by the project and the expectations of the software.


\subsection{Document Conventions}
This document was made using \LaTeX

\subsection{Intended Audience}
This SRD is intended for Professor Ike, and the rest of the class.

\subsection{Definitions/Jargon}

SRD : Software Requirements Document

JRE : Java Runtime Environment

JDK : Java Development Environment

JVM : Java Virtual Machine 

\subsection{Project Scope}
The scope of this project consists of creating cards, creating events, using the google calendar API to store the events externally, and being able to view said events.

\subsection{Technical Challenges}
JavaFX proved to be tough, along with figuring out the best way to store cards, along with reading and writing to them.
\subsubsection{Github and Git}
Github became a challenge in itself as the group hadn't used it before. Figuring out how to make commits, and how to keep the files the same in the group was a continuous challenge.

\subsection{References}

\section{Overall Description}
\subsection{Product Features}
The product allows you to create cards, and in them create divided subsections. You can name the cards, then name the individual subsections within them. You can create events in the cards and they will appear in your Google Calendar. You are also able to view the events you have created through the app in the Calendar view.

\subsection{User Characteristics}
The user is not required to have any past experience with applications to operate the app.

\subsection{Operating Enviornment}
This software is intended to be used in a professional environment.

\subsection{Design and Implementation Constraints}

This software is required to be implemented in Java. 
\subsection{Assumptions and Dependencies}
This product depends on JavaFX and it also depends on the Jsoup library. It comes prepackaged in the app. It is assumed the system running it is capable of running both.

\section{Functional Requirements}

\subsection{Primary Functions}
The primary function of the application is to be able to create cards. This feature will always be available as it does not depend on outside sources.
\subsection{Secondary Functions}
Creating and viewing tasks may not always be available as it relies on an external api.
\section{Technical Requirements}
\subsection{Operating Systems/Compatibility}

This software will use libraries that are cross-platform to some extent
to be allowed to work on operating systems that run JRE 8.

\subsection{Interface Requirements}

\subsubsection{User Interface}\label{user-interface}

The User Interface must not be cluttered, and needs to be intuitive.

\subsubsection{ Hardware Interface}
The software will not require any special hardware interfaces beyond what is required of a standard Java Desktop Application which includes but not limited to the following:
\begin{itemize}
\item A Monitor
\item CPU w/ multithreading capabilities
\item RAM
\item Storage Device
\end{itemize}

\subsubsection{ Software Interface}
This software will require \acrshort{jre} 8, and some graphical service (like Xorg on linux).
\subsubsection{Communications Interface}
This software will use an API on a remote service to \ldots
\section{Nonfunctional Requirements}

\subsection{Performance Requirements}

\subsection{Safety/Recovery Requirements}

The software will have the following features to protect the user data: (1) Backup saving system to prevent primary file from being corrupted due to premature shutdown. (2) Autosaving sytem to save userdata a change has been made. 

\subsection{Security Requirments}
The software would normally have security requirements to protect its users. However for this project, it is beyond its scope and will not be worked on. 

\subsection{Policy Requirements}

\subsection{Software Quality Attributes}

\subsubsection{Availability}
This software will have the following feature(s) that will become unavailable without Internet access:
\begin{itemize}
\item Google Calendar (Syncing)
\end{itemize}
\subsubsection{Correctness}
This software will strive to ensure that data is properly saved to prevent loss of the aforementioned data.
\subsubsection{Maintainability}
This software will strive to follow the standards for object oriented programming to make it easier to maintain the software.

\subsubsection{Re-usability}
 This software will strive to follow the standards for object oriented programming. 
\subsubsection{Portability }
This software will strive to work cross-platform. 

\subsection{Process Requirements}

\subsubsection{Development Process Used}

Using the AGILE development method, we had weekly meetings to discuss and share what we had planned. Work was assigned and ideas were shared.

\subsubsection{Time Constraints}

There are no time constraints except for the delivery date.
\subsubsection{ Cost and Delivery Date}

 The delivery date for the project is
\textbf{\emph{December 1, 2020}}. There is no expected cost for development.

\end{document}
